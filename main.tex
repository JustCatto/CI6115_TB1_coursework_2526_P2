%! Author = Romvulus Casunuran
%! Date = 04/01/2026

% Preamble
\documentclass[11pt]{article}

\usepackage{listings}
\usepackage{color}
\usepackage{graphicx}
\usepackage{enumitem}
\usepackage[backend=biber,style=authoryear]{biblatex}
\usepackage{float}
\usepackage{array}


\graphicspath{{./img/}}
\addbibresource{CL6115_TB1_coursework_2526.bib}

\definecolor{dkgreen}{rgb}{0,0.6,0}
\definecolor{gray}{rgb}{0.5,0.5,0.5}
\definecolor{mauve}{rgb}{0.58,0,0.82}

\newcolumntype{M}[1]{>{\centering\arraybackslash}m{#1}}

\lstset{frame=tb,
  language=Java,
  aboveskip=3mm,
  belowskip=3mm,
  showstringspaces=false,
  columns=flexible,
  basicstyle={\small\ttfamily},
  numbers=none,
  numberstyle=\tiny\color{gray},
  keywordstyle=\color{blue},
  commentstyle=\color{dkgreen},
  stringstyle=\color{mauve},
  breaklines=true,
  breakatwhitespace=true,
  tabsize=3
}


% Document
\begin{document}
  \section{Question 1:}\label{sec:question-1:}
  \subsection{Book.java:}\label{subsec:book.java}
  \lstinputlisting[language=Java,label={lst:book}]{src/main/java/Book.java}
  \newpage
  \subsection{Item.java:}\label{subsec:item.java}
  \lstinputlisting[language=Java,label={lst:item}]{src/main/java/Item.java}
  \newpage
  \subsection{Magazine.java:}\label{subsec:magazine.java}
  \lstinputlisting[language=Java,label={lst:magazine}]{src/main/java/Magazine.java}
  \newpage
  \subsection{Notebook.java:}\label{subsec:notebook.java}
  \lstinputlisting[language=Java,label={lst:notebook}]{src/main/java/Notebook.java}
  \newpage
  \subsection{Person.java:}\label{subsec:person.java}
  \lstinputlisting[language=Java,label={lst:person}]{src/main/java/Person.java}
  \newpage
  \subsection{ProductPhoto.java:}\label{subsec:productphoto.java}
  \lstinputlisting[language=Java,label={lst:productphoto}]{src/main/java/ProductPhoto.java}
  \section{Question 2:}\label{sec:question-2}
  \includegraphics[width=\linewidth]{q2}
  \section{Question 3:}\label{sec:question-3}
  \subsection{a)}\label{subsec:3a)}
  Provide the general UML Class diagram for the GoF Factory Method Pattern.
  You can use an existing diagram from a source (e.g.\ teaching material,
  wikipedia, book, etc), but clearly cite the source.
  (0.5 mark)
  \begin{figure}[ht]
    \includegraphics[width=\linewidth]{gof_example}
    \caption{An example of the GoF Factory Method Pattern \parencite{w3sdesignGoFDesignPatterns}}
    \label{fig:gof_example}
  \end{figure}
  \subsection{b)}\label{subsec:3b)}
  Apply the GoF Factory Method Pattern on the Class Diagram of Q1. The
  purpose is to satisfy the requirement of creating different types of items
  dynamically.
  Show your solution as a UML Class Diagram that exemplifies the
  UML general Class diagram you provided in Q3a on the system of Q1.
  (2 marks)
  \begin{figure}[ht]
      \includegraphics[width=\linewidth]{q3}\label{fig:3b-diagram}
  \end{figure}
  \subsection{c)}\label{subsec:3c)}
  Make clear the correspondence between the pattern concepts as shown in the
  UML general Class diagram you provided in Q3a and the
  interfaces/classes/methods/attributes of the system shown in Q1. 
  (1.5 mark)\newline
  See figure~\ref{fig:q3-code}
  \begin{table}[H]
    \centering
    \begin{tabular}{M{0.5\textwidth}|M{0.5\textwidth}}
      Part & Class Name \\ \hline
      Creator (Creator) & Factory \\ \hline
      Concrete Creator (Creator1) & \begin{itemize}
                                      \item BookFactory
                                      \item NotebookFactory
                                      \item MagazineFactory
      \end{itemize} \\ \hline
      Product (Product) & Item \\ \hline
      Concrete Product (Product1) & \begin{itemize}
                                       \item Notebook
                                       \item Book
                                       \item Magazine
      \end{itemize} \\
    \end{tabular}
    \label{tab:q3-mapping}
  \end{table}

  \section{Question 4:}\label{sec:question-4:}
  \subsection{a)}\label{subsec:4a}
  \begin{figure}[ht]
    \includegraphics[width=\linewidth]{q4a}
    \caption{A diagram of the GoF Observer Pattern \parencite{ObserverPattern2025}}
    \label{fig:q4a}
  \end{figure}

  \subsection{b and c)}\label{subsec:4b-and-c}
  \begin{figure}[H]
    \includegraphics[width=\linewidth]{q4b}
    \label{fig:q4b}
  \end{figure}

  \subsection{d}\label{subsec:4d}
  \begin{table}[H]
    \centering
    \begin{tabular}{M{0.5\textwidth}|M{0.5\textwidth}}
      Part & Class Name \\ \hline
      Observer & Observer \\ \hline
      ConcreteObserverA/B & \begin{itemize}
                              \item EmailSystem
                              \item InventoryManagement
                              \item OrderTracking
      \end{itemize}\\ \hline
      Subject & Observable \\ \hline
      ConcreteSubject & OrderSystem \\
    \end{tabular}
    \label{tab:q4-mapping}
  \end{table}

  \section{Question 5:}\label{sec:question-5}
  \subsection{a)}\label{subsec:5a}
  Provide background information about your project (e.g.\ topic, programming language, platform, etc) (1 mark).
  \newline
  The project is an implementation of Othello with a minimax AI in python, designed to run in a CLI\@.

  \subsection{b}\label{subsec:5b}
  Explain the problem (antipattern/code smell) in your project, using artefacts (code, architecture, management) that show your project before applying
  refactoring (3 marks).
  As seen in figure~\ref{subsec:figure-5-code}, this code has a couple of code smells.
  \begin{itemize}
    \item Excessive Comments
    \item Primitive Obsession
    \item Duplicate Code
    \item Feature Envy
    \item Long Method
    \item Large Class
    \item
  \end{itemize}

  \subsection{c}\label{subsec:5c}
  Discuss the refactoring solution you applied and provide details on how it was applied to your project.
  Include any artefacts (code, diagrams, processes) altered by the refactoring solution you applied (3 marks).

  \section{Appendix}\label{sec:appendix}
  \begin{enumerate}[label=\alph*)]
    \item Software Tool used for drawing UML Diagrams: StarUML
    \item {Did you use software to automate conversion of diagrams to code or code diagrams?
    \begin{itemize}
      \item {
        YES, I USED software to automate conversions of diagrams to code and/or
      code to diagrams
      \begin{itemize}
              \item if YES, complete software name here: StarUML Java Reverse Engineering
      \end{itemize}
      }
    \end{itemize}
    }
    \item {Did you use Generative AI for this Submission? (delete as appropriate)
    \begin{itemize}
      \item NO, Generative AI was NOT USED for this submission.
    \end{itemize}
    }
  \end{enumerate}

  \subsection{Q3 Code}\label{subsec:q3-code}
  \lstinputlisting[language=Java,label={lst:q3}]{src/main/java/CopyPasta.java}\label{fig:q3-code}

  \subsection{Figure 5 code}\label{subsec:figure-5-code}
  \subsubsection{Player}
  \lstinputlisting[language=Python,label={lst:figure-5-player-py}]{python/src/player.py}
  \subsubsection{Board}
  \lstinputlisting[language=Python,label={lst:figure-5-Board-py}]{python/src/board.py}

  \subsection{Figure 5 code refactored}\label{subsec:figure-5-code-refactored}

  \printbibliography
\end{document}